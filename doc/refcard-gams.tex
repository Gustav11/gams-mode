%#!pdflatex 
% Reference Card for GAMS mode.

% Time-stamp: <2018-04-16 18:08:14 st>

% This file is based on gnusref.tex distriubted with T-Gnus.  Shiro Takeda
% has modified it and created this file.

% Use latex2e.
\documentclass{article}

\usepackage{mathpazo}

% Layout.
\textwidth 6.66in \textheight 10.4in \topmargin -1.1in

% the same settings work for A4, although there is a bit of space at the
% top and bottom of the page.
\oddsidemargin -0.2in \evensidemargin -0.3in

\begin{document}

% Width between two columns.
\setlength{\columnsep}{1cm}

% Two columns, footnote size.
\twocolumn\footnotesize\pagestyle{empty}

\def\version{6.5}
\def\date{April, 2018}
\def\year{2001--2018\ }

\def\author{Shiro Takeda \\{\tt shiro.takeda@gmail.com}}

\raggedbottom\raggedright

\newenvironment{keys}[1]% #1 is the widest key
  {\noindent
%   {\nopagebreak%\noindent%
   \settowidth{\keycolwidth}{#1}%
   \addtolength{\keycolwidth}{\tabcolsep}%
   \addtolength{\keycolwidth}{-\columnwidth}%
   \begin{tabular}{@{}l@{\hspace{\tabcolsep}}p{-\keycolwidth}@{}}}%
  {\end{tabular}\\}

\newdimen\defwidth \defwidth=0.25\hsize
\def\hang{\hangindent\defwidth}
\def\textindent#1{\noindent\llap{\hbox to \defwidth{\tt#1\hfil}}\ignorespaces}
\def\key{\par\hangafter=0\hang\textindent}

\font\headingfont=cmss10
\def\sec#1{\par
  \vskip 3ex plus 2ex minus 2ex
  {\headingfont #1}\mark{#1}%
  \vskip 2ex plus 1ex minus 1.5ex}

\catcode`\^=12 % allow ^ to be typed literally
\catcode`\~=12 % allow ~ to be typed literally

\newcommand{\B}[1]{{\bf#1})}    % bold l)etter
\newcommand{\kbd}[1]{{\tt#1}}

\def\cr{
\begin{center}
Copyright \copyright \ \year \ \author.\vspace*{1em}\\

Permission is granted to make and distribute copies of this card
provided the copyright notice and this permission notice are preserved
on all copies.
\end{center}
}


%%% Title.

\begin{center}
 {\sf
 {\large Reference Card for GAMS mode} \vspace*{0.5em}\\
 {\small for gams-mode.el version \version, \date.}
 }
\end{center}

%%%%%% Notes

\sec{Notes}
In the following, \kbd{C-c} means hit the `\kbd{c}' key while holding down
the \kbd{Ctrl} key. \kbd{M-z} means hit the `\kbd{z}' key while hitting
the \kbd{Meta} (labeled \kbd{Alt} on some keyboards) or after hitting
\kbd{Esc} key.  The slash `\kbd{/}' means `or'.  Of course, these are
the {\it default} key-bidndings and you can change them freely.

%%%%%% GAMS mode.

\sec{GAMS mode}

\key{C-c C-k} Insert GAMS statement.
\key{C-u C-c C-k} Replace the existing statement with new one.
\key{C-c C-d} Insert GAMS dollar control.
\key{C-u C-c C-d} Replace the existing dollar control with new one.
\key{C-c C-n} Insert GAMS statement (extended).
\key{C-c C-o} Insert a comment template.
\key{C-c C-e} Start the GAMS-TEMPLATE mode.
\key{C-c C-w} Open the included subroutine file.
\key{} 
\key{C-c C-v/F10} Switch to the LST file and show errors.
\key{C-c C-j} Switch to the LST file.
\key{C-c C-i/F11}  Switch to the OUTLINE buffer.
\key{} 
\key{C-c C-t}  Start GAMS process menu.
\key{C-c C-s/F9}  Run GAMS.
\key{C-c C-l}  Popup GAMS process buffer.
\key{} 
\key{C-c C-c}  Insert an ontext-offtext pair.
\key{C-c C-g}  Jump between an ontext-offtext pair.
\key{C-c M-c}  (Un)comment an ontext-offtext pair.
\key{C-c M-g}  Remove an ontext-offtext pair.
\key{} 
\key{(C-u)C-cC-.}  Show the declaration part of an identifier.
\key{\ \ \ \ \ \ or F7}
\key{C-c C-a}  Show the list of all identifiers in the buffer.
\key{F8}  Jump between the matched parenthesis.
\key{C-c C-f}  Choose font-lock level.
\key{M-g M-g}  Fontify block.
\key{C-l}  Recentering and recoloring.
\key{TAB}  Indent line.
\key{} Indent region is \verb|`M-C-\'|
\key{C-c C-m}  View GAMS pdf manuals.
\key{C-c C-z}  View GAMS model library.
\key{M-;}  Insert end-of-line comment.
\key{C-c M-;}  Insert inline comment.
\key{C-c C-;}  Comment out a region.
\key{C-c C-h}  Hide (show) comment region.
\key{C-c C-x}  Start GAMS-LXI mode.
\key{C-c C-y}  Align blocks.

%%%%%%

\sec{GAMS--LST mode}

\key{y}   Jump to the error and show its number and meaning.
\key{u}  Jump back to the error place in the program file.
\key{i}  Jump to the input (gms) file.
\key{o}  Start the GAMS-OUTLINE mode.
\key{.}  Show the Included File Summary
\key{} 
\key{q}  Close the buffer.
\key{?}  Display the help.
\key{} 
\key{s(S)}  Jump to the next (previous) SOLVE SUMMARY.
\key{r(R)}  Jump to the next (previous) REPORT SUMMARY.
\key{v(V)}  Jump to the next (previous) VAR entry.
\key{e(E)}  Jump to the next (previous) EQU entry.
\key{p(P)}  Jump to the next (previous) PARAMETER entry.
\key{x(X)}  Jump to the next (previous) Equation Listing entry.
\key{c(C)}  Jump to the next (previous) Column Listing entry.
\key{L}  Jump to a line you specify.
\key{l}  Jump to a line.
\key{} 
\key{SPC}  Scroll up.
\key{M-v/DEL}  Scroll down.
\key{1}  Widen the window.
\key{2}  Split the window.
\key{m}  Move frame.
\key{w}  Resize frame.
\key{z}  Move a cursor to the other window.
\key{} 
\key{d,f,g,h,j,k}          Scroll commands.

%%%%%%

\sec{GAMS--OUTLINE mode}

\key{SPC}  Show the content of the item on the current line.
\key{N}  Show the content of the next item.
\key{P}  Show the content of the previous item.
\key{t}  Select viewable items.
\key{T}  Select registered viewable item combination.
\key{c}  Toggle the follow mode.
\key{x}  Toggle the display style.
\key{} 
\key{m}  Mark an item.
\key{u}  Unmark an item.
\key{y}  Jump to the marked line.
\key{} 
\key{i}  Switch back to the LST buffer.
\key{;}  Switch back to the gms file buffer.
\key{?}  Show the help.
\key{q}  Quit.
\key{} 
\key{l}  Widen the window with one line.
\key{o}  Narrow the window with one line.
\key{1}  Widen the window.
\key{C-l}  Recenter.
\key{C-v/ENT}  Scroll up the OUTLINE buffer.
\key{M-v/DEL}  Scroll down the OUTLINE buffer.
\key{} 
\key{n}          Next line.
\key{p}          Previous line.
\key{w}          Resize frame.
\key{e}          Move frame.
\key{} 
\key{d,f,g,h,j,k}          Scroll commands.

%%%%%%

\sec{GAMS--TEMPLATE mode}

\key{SPC}  Show the content of a template.
\key{ENT}  Insert a template.
\key{a}  Add a new template.
\key{d}  Delete a template.
\key{e}  Re-edit a template.
\key{r}  Rename a template.
\key{u}  Move a template up.
\key{j}  Move a template down.
\key{g}  Show the gms file.
\key{q}  Quit.
\key{} 
\key{s(S)}       Scroll up (down) *Template Content* buffer.
\key{h}  Show the help.
\key{o}  Save the content of gams-user-template-alist.
\key{p}  Show the previous template.
\key{n}   Show the next template.


%%%%% Copyright.

\vspace*{\fill}
\cr

\end{document}
